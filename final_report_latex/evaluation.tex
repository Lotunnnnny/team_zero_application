The Team Zero Application has overall met most of the criteria it has set out to meet. It is a working distributed chat system that supports the sending/receiving of encrypted messages between clients, having two different client types (Android and web) and allows for expansion to other client types due to the nature of the Server API. These achievements are generally summarised in section \ref{summary_achievements}.

As a team, we agreed to meet weekly and had each our area of ``expertise". We were able to stick to these most of the time, and helped each other out more towards the end. Although we had some previous knowledge in a few areas of software development, we each still had to research a lot and learn new technologies to be able to build the Team Zero Application system. This allowed for each team member to gain at least some understanding of each area of the system. There were times when we were unable to meet weekly, but made up for it by keeping in touch via social media (WhatsApp) so there were no communication gaps. We believe we worked extremely well together, although the schedule in which each of us were able to commit to large amounts of time to the project varied due to our different course programs and alternating workloads in other modules. Although this was a challenge however, it did not affect our overall teamwork and ability to work on the project together, as we maintained effective communication throughout.

We underestimated the difficulty of implementing end-to-end encryption, which significantly shifted our schedule and affected our ability to fully implement group chats. The main issue was interoperability between encryption protocols in the libraries that we tried to use, as they used low level cryptographic algorithms with differing default parameters. Other deviations from requirements, due to time constraints and due to prioritisation are those written at the start of section \ref{deviations-from-requirements}. We believe the skipped requirements to not be essential to the functioning of the system, and therefore not a great loss. However, it would have been better to have been able to fully implement group chat on both clients with encrypted messages - as currently only the web client supports group chats, and the messages are not encrypted. Since the Server API contains the appropriate tools for it, time is the only resource required to be able to complete the full implementation of the group chat feature. 


Another potential weakness of the system is that while there is end-to-end encryption for client messages, it is only for the messages themselves. The metadata that surrounds the messages is not encrypted between client and server. Therefore the security of encrypted messages is slightly compromised by the usernames of sender and recipients and timestamps of messages being visible, as well as information such as login data and other requests to the server. However, there is an easy fix to this, which is to use the secure websocket protocol over SSL, as currently the system just uses the websocket protocol. This secure protocol was not used for this project as it was important for metadata messages to and from the server to be visible for debugging and testing purposes.


The implementation of encryption itself has some weaknesses as well. As mentioned in section \ref{encryption}, the public and private keys are not currently generated as they should be by the DH Key generation, due to interoperability issues. This is seen as a temporary issue, or rather a temporary fix put in place to be able to demonstrate that the rest of the encryption protocols could work and perform well. Additionally, group messages are not encrypted, but several methods by which encryption can be implemented in the future is outlined in section ref{encryption-groups}. With more time and resources, we believe both group encryption and regular chat encryption could be fully implemented as described in \ref{encryption}.

One point that will be important to consider in the future is that currently the system allows for clients to search for and view all other users/clients registered with Team Zero Chat Application. While we believe this is fine for a small system user base, it would become problematic to navigate as the number of users increase, and as users may not wish to be visible to people they do not personally know. 

In regards to the tests, we have created several scenarios to follow as strictly as possible after each meeting conducted every week on Monday. From registration of new users, logging back in and chatting between clients to ensuring that the server is always passing the right information and the clients process them equally correct. There have been times when we found various bugs in our system and 

In conclusion, we believe that while the project has had its issues, the system comes out with many strengths overall. In particular, we wish to draw attention to the functional server API, which is easy to extend and allows for multiple client implementations regardless of device, so long as messages are sent in the right format - therefore this system can be extended to other client implementations in the future. 
